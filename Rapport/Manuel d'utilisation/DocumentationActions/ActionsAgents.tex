	\subsubsection{A\_GiveBirth} 
	
	L'action permet à l'agent qui l'exécute de créer un nouvel agent sur le patch où il se situe : cet agent sera identique à ceux du peuplement initial.
	
	\subsubsection{A\_Die}
	
	L'action permet à l'agent qui l'exécute de se supprimer.
	
	\subsubsection{A\_DieIfAttributeUnderZero}
	
	L'action supprime l'agent qui l'exécute si la valeur de l'attribut (ou caractéristique) passé en paramètre descend en dessous de zéro. La signature de cette action est A\_DieIfAttributeUnderZero(attributeToCompare) :
	
	
	\begin{itemize}
	\item \texttt{attributeToCompare}, correspond à l'attribut à vérifier
	\end{itemize}	
	
	\subsubsection{A\_ChangeAttribute} 
	
	L'action modifie une caractéristique de l'agent. 
	
	La signature de l'action est A\_ChangeAttribute(Modified attribute, n) ;
	
	\begin{itemize}
	\item \texttt{Modified attribute}, correspond à la caractéristique à modifier
	\item \texttt{n}, donne la valeur à ajouter à la caractéristique
	\end{itemize}
	
	\subsubsection{A\_Trade} 
	
	L'action A\_Trade  comporte deux actions internes, elle correspond à un échange conditionnel,  car elle suppose que si après un temps fixé (\textit{turns}) l'agent n'a pas réussi à trouver un partenaire il exécute 
la deuxième action interne sinon la première.

	La signature de l'action est A\_Trade(turns, objectToGive, nObjectToGive, objectToTake, nObjectToTake, myTag, compatibleTag) ;
	
		\begin{itemize}
	\item \texttt{turns}, correspond au nombre de ticks durant lequel l'agent attend sur son patch un autre agent pour réaliser un échange,
	\item \texttt{objectToGive}  donne l'objet que l'agent échange,
	\item\texttt{nobjectToGive} correspond au nombre n d'objets à échanger,
	\item \texttt{objectToTake}  précise l'objet reçu en échange,
	\item \texttt{nobjectToTake} correspond au nombre n d'objets reçus,
	\item \texttt{mytag}, message que l'agent passe à l'attention des autres,
	\item \texttt{compatibletag} message recherché chez le partenaire échangeur potentiel.
	\end{itemize}
	
	\subsubsection{A\_TravelTrade} 
	
	L'action A\_TravelTrade  comporte deux actions internes, elle correspond à un échange conditionnel,  car elle suppose que si après un temps fixé (\textit{turns}) l'agent n'a pas réussi à trouver un partenaire il exécute la deuxième action interne sinon la première.

	La signature de l'action est A\_TravelTrade(turns, objectToGive, nObjectToGive, objectToTake, nObjectToTake, myTag, compatibleTag) ;
	
		\begin{itemize}
	\item \texttt{turns}, correspond au nombre de ticks durant lequel l'agent cherche des partenaires commerciaux dans son voisinage et se déplace vers eux,
	\item \texttt{objectToGive}  donne l'objet que l'agent échange,
	\item\texttt{nobjectToGive} correspond au nombre n d'objets à échanger,
	\item \texttt{objectToTake}  précise l'objet reçu en échange,
	\item \texttt{nobjectToTake} correspond au nombre n d'objets reçus,
	\item \texttt{mytag}, message que l'agent passe à l'attention des autres,
	\item \texttt{compatibletag} message recherché chez le partenaire échangeur potentiel.
	\end{itemize}
	
	
	\subsubsection{A\_CreateGroup}
	
	L'action permet à l'agent  de créer un groupe. 
	
	La signature de l'action  est A\_CreateGroup(GroupToCreate) : 
	
	\begin{itemize}
	\item \texttt{GroupToCreate}, donne le type du groupe à créer.
	\end{itemize}
	
	\subsubsection{A\_HireForRole}
	
	L'action permet à l'agent exécutant d'ajouter un agent présent sur le même patch que lui à son groupe . 
	
	La signature de l'action est A\_HireForRole(GroupToCreate) :
	
	\begin{itemize}
	\item \texttt{GroupToCreate}, donne  le type du groupe dans lequel  l'agent sera admis
	\end{itemize}
	
	\textit{Remarque : un agent peut appartenir à plusieurs groupes}
	
	
	\subsubsection{A\_DisbandGoup}

Cette action détruit un groupe et retire tout les membres de celui-ci
La signature de l'action est A\_DisbandGoup(Groupe) :
\begin{itemize}
\item Groupe, le nom du groupe auquel appartient l'agent qui détruit le groupe.
\end{itemize}

\subsubsection{A\_CreatedElectedGroup}

Cette action a pour but la création d'un groupe a travers un vote. Si 50\% des votants approuve la motion, alors tout les votant rejoingne le groupe dans un certain role et le demander du vote dans un autre. 
La signature de l'action est A\_CreatedElectedGroup(Role, Role, cogniton, attribut) :
\begin{itemize}
\item Role, le nom du role dont le demandeur va rejoindre.
\item Role, le nom du role dont les votant vont rejoindre.
\item cogniton, le cogniton surlequel se base le vote.
\item attribut, le nom de l'attribut qui conditionnent le vote des votant.\end{itemize}


\subsubsection{A\_HireForRole}

Cette action recrute un agent sans groupe se trouvant sur le même patch que l'agent et lui donne un rôle dans son groupe.
La signature de l'action est A\_HireForRole(Role) :
\begin{itemize}
\item Role, le rôle dans lequel le nouvel agent recruté sera affecté dans le groupe.
\end{itemize}

\subsubsection{A\_AskRandomMemberToChangeRoleForAnother}

Cette action change le rôle d'un des membre du groupe de l'agent pour un autre rôle spécifié (s'il n'est pas déjà dans ce rôle).
La signature de l'action est A\_AskRandomMemberToChangeRoleForAnother(Role):
\begin{itemize}
\item Role, le rôle vers lequel un membre du groupe se convertira.
\end{itemize}

\subsubsection{A\_BirthGroupAndRole}

Cette action ressemble à l'action A\_Birth, elle créer un nouvel agent en appelant le BirthPlan puis ajoute le nouveau membre au groupe en lui affectant un rôle.
La signature de l'action est A\_BirthGroupAndRole(Role):
\begin{itemize}
\item Role, le rôle vers lequel le nouvel agent sera affecté dans le groupe.
\end{itemize}

\subsubsection{A\_ChangeAttributeDouble}

Cette action ressemble à l'action A\_ChangeAttribute, elle permet de modifier l'attribut d'un agent en ajoutant un N (double) à la valeur de l'attribut.
La signature de l'action est A\_ChangeAttributeDouble(Modified Attribute, N):
\begin{itemize}
\item Modified Attribute, attribut de l'agent à modifier.
\item N, la valeur à ajouter à la valeur de l'attribut.
\end{itemize}

\subsubsection{A\_ChangeRoleForAnother}

Cette action change le rôle de l'agent pour un autre rôle du groupe.
La signature de l'action est A\_ChangeRoleForAnother(Role):
\begin{itemize}
\item Role, le rôle vers lequel l'agent se convertira.
\end{itemize}

\subsubsection{A\_DieAndRemoveFacilities}

Cette action fait mourir l'agent et efface tous ses aménagements.
Cette action ne possède pas d'arguments.

\subsubsection{A\_DieAndRemoveSpecificFacility}

Cette action fait mourir l'agent et n'efface qu'un type d'aménagement.
La signature de l'action est A\_DieAndRemoveSpecificFacility(Facility):
\begin{itemize}
\item Facility, le type d'aménagements supprimés à la mort de l'agent.
\end{itemize}


\subsubsection{A\_setPoissonLaw}

Cette action modifie un attribut et lui donne la valeur d'un nombre suivant un loi de poisson de paramètre $\lambda = 10$ et $n=20$ et multiplié par 5.
La signature de l'action est A\_setPoissonLaw(Modified Attribute):
\begin{itemize}
\item Modified Attribute, est l'attribut modifié.
\end{itemize}