

La structure générale des actions logiques de type Test est :

\begin{algorithm}
\Begin{
       
    \If{ condition }
  			 { 
  		 		action interne 1 
  		 		
   			}	   
        	\Else
        	{
        		action interne 2 
        		    	
        	}
   }  
  
\end{algorithm}

\newpage
\textit{Remarque : ces actions logiques sont toutes composées de deux actions internes (qui peuvent elles-mêmes être composites).}

Les diverses actions suivantes vont préciser le type de condition.

\subsubsection{L\_CompareAttribute}
La condition porte sur la valeur d'un attribut de l'agent.

La signature de cette action est L\_CompareAttribute(attributTocompare, comparator, n) :
\begin{itemize}
	\item \texttt{attributTocompare}, est l'attribut concerné,
	\item \texttt{comparator} l'opérateur de comparaison $<$, $>$, $<=$,$ >=$,$ ==$,
	\item \texttt{n} la valeur de comparaison.
	\end{itemize}

\subsubsection{L\_CompareObject}
La condition porte sur le nombre d'objets d'un type donné possédés.

La signature de cette action est L\_CompareObject(objectTocompare, comparator, n) :
\begin{itemize}
	\item \texttt{objectTocompare}, est le type d'objet concerné,
	\item \texttt{comparator} l'opérateur de comparaison $<$, $>$, $<=$,$ >=$,$ ==$,
	\item \texttt{n} le nombre d'objets.
	\end{itemize}
\subsubsection{L\_CompareResource}
La condition porte sur la valeur de la ressource présente sur le patch où est situé  l'agent.

La signature de cette action est L\_CompareResource(resourceTocompare, comparator, n) :
\begin{itemize}
	\item \texttt{resourceTocompare}, est le type de la ressource concernée,
	\item \texttt{comparator} l'opérateur de comparaison $<$, $>$, $<=$,$ >=$,$ ==$,
	\item \texttt{n} la valeur de la ressource.
	\end{itemize}
	
	
\subsubsection{L\_IsFacilityHere}

La condition teste la présence de l'aménagement \textit{(facility)} concerné sur le patch  où est situé  l'agent.

La signature de cette action est L\_IsFacilityHere(facility) :
\begin{itemize}
	\item \texttt{facility},  le type d'aménagement testé.
	\end{itemize}	
	
\subsubsection{L\_IsAnyFacilityHere}
La condition teste la présence d'un aménagement de n'importe quel type sur le patch  où est situé  l'agent.

\subsubsection{L\_OwnFacility}

La condition teste la possession d'un aménagement donné.

La signature de cette action est L\_OwnFacility(facility) :
\begin{itemize}
	\item \texttt{facility},  le type d'aménagement testé.
	\end{itemize}	
\subsubsection{L\_OwnCogniton}
La condition teste la possession d'un cognition.

La signature de cette action est L\_OwnCogniton(cogniton) :
\begin{itemize}
	\item \texttt{cogniton},  le type de cogniton testé.
	\end{itemize}	
\subsubsection{L\_OwnObject}

La condition teste la possession d'un objet. 

La signature de cette action est L\_OwnObject(object) :
\begin{itemize}
	\item \texttt{object},  le type d'objet testé.
	\end{itemize}	

\subsubsection{L\_IsAmenagementInCommunity}

La condition porte sur la présence d'un aménagement dans la civilisation. Il s'agit de tester si ce type d'aménagement est présent.
La signature de cette action est L\_CompareAttributeToAttribute(TypeAmenagement):
\begin{itemize}
\item TypeAmenagement est le type d'aménagement attribut concerné.
\end{itemize}


\subsubsection{L\_CompareAttributeToAttribute}

La condition porte sur les valeurs de deux attributs de l'agent, elles sont comparés par un comparateur.
La signature de cette action est L\_CompareAttributeToAttribute(Attribute1, comparator, Attribute2):
\begin{itemize}
\item Attribute1 est le premier attribut concerné.
\item comparator est l'opérateur de comparaison $<, >, \leq, \geq, == $.
\item Attribute2 est le deuxième attribut concerné.
\end{itemize}

\subsubsection{L\_CompareAttributeToConstant}

La condition porte sur la valeur d'un attribut par rapport à une constante.
La signature de cette action est L\_CompareAttributeToConstant(Attribute, comparator, Constant):
\begin{itemize}
\item Attribute est l'attribut concerné.
\item comparator est l'opérateur de comparaison $<, >, \leq, \geq, == $.
\item Constant est la constante concernée.
\end{itemize}

\subsubsection{L\_CompareGroupAttributeToPopulation}

La condition porte sur un la somme des attributs des membres d'un groupe par rapport au nombre d'individus de ce groupe.
La signature de cette action est L\_CompareGroupAttributeToPopulation(Attribute, comparator):
\begin{itemize}
\item Attribute, est l'attribut de groupe à comparer.
\item comparator est l'opérateur de comparaison $<, >, \leq, \geq, == $.
\end{itemize}

%L\_CompareNombreObjets
%L\_CompareNombreObjetsInGroupAmenagement

\subsubsection{L\_CompareRoleMembers}

La condition porte sur le nombre d'individus du groupe par rapport à un nombre N.
La signature de cette action est L\_CompareRoleMembers(N, comparator):
\begin{itemize}
\item N est le nombre à comparer à la population du groupe.
\item comparator est l'opérateur de comparaison $<, >, \leq, \geq, == $.
\end{itemize}

\subsubsection{L\_IsInThatGropInThatRole}

La condition porte sur le rôle de l'agent dans son groupe.
La signature de cette action est L\_IsInThatGropInThatRole(Role):
\begin{itemize}
\item Role, le rôle à comparer avec le rôle de l'agent.
\end{itemize}
